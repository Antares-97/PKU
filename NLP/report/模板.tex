\documentclass[paper=a4, fontsize=11pt]{scrartcl} % A4 paper and 11pt font size
\usepackage[UTF8, heading = false, scheme = plain]{ctex}
\usepackage[T1]{fontenc} % Use 8-bit encoding that has 256 glyphs
\usepackage{fourier} % Use the Adobe Utopia font for the document - comment this line to return to the LaTeX default
\usepackage[english]{babel} % English language/hyphenation
\usepackage{amsmath,amsfonts,amsthm} % Math packages

\usepackage{lipsum} % Used for inserting dummy 'Lorem ipsum' text into the template

\usepackage{sectsty} % Allows customizing section commands
\allsectionsfont{\centering \normalfont\scshape} % Make all sections centered, the default font and small caps

\usepackage{fancyhdr} % Custom headers and footers

\usepackage{listings}



\pagestyle{fancyplain} % Makes all pages in the document conform to the custom headers and footers
\fancyhead{} % No page header - if you want one, create it in the same way as the footers below
\fancyfoot[L]{} % Empty left footer
\fancyfoot[C]{} % Empty center footer
\fancyfoot[R]{\thepage} % Page numbering for right footer
\renewcommand{\headrulewidth}{0pt} % Remove header underlines
\renewcommand{\footrulewidth}{0pt} % Remove footer underlines
\setlength{\headheight}{13.6pt} % Customize the height of the header

\numberwithin{equation}{section} % Number equations within sections (i.e. 1.1, 1.2, 2.1, 2.2 instead of 1, 2, 3, 4)
\numberwithin{figure}{section} % Number figures within sections (i.e. 1.1, 1.2, 2.1, 2.2 instead of 1, 2, 3, 4)
\numberwithin{table}{section} % Number tables within sections (i.e. 1.1, 1.2, 2.1, 2.2 instead of 1, 2, 3, 4)

\setlength\parindent{0pt} % Removes all indentation from paragraphs - comment this line for an assignment with lots of text

%----------------------------------------------------------------------------------------
%	TITLE SECTION
%----------------------------------------------------------------------------------------

\newcommand{\horrule}[1]{\rule{\linewidth}{#1}} % Create horizontal rule command with 1 argument of height

\title{	
\normalfont \normalsize 
\textsc{Peking University, Department of Physics} \\ [25pt] % Your university, school and/or department name(s)
\horrule{0.5pt} \\[0.4cm] % Thin top horizontal rule
\huge 基于双向LSTM的sqe2seq中文分词 \\ % The assignment title
\horrule{2pt} \\[0.5cm] % Thick bottom horizontal rule
}

\author{汪子龙,1500011371} % Your name

\date{\normalsize\today} % Today's date or a custom date

\begin{document}

\maketitle % Print the title

%----------------------------------------------------------------------------------------
%	PROBLEM 1
%----------------------------------------------------------------------------------------

\section{问题描述}

%\lipsum[2] % Dummy text

% \begin{align} 
% \begin{split}
% (x+y)^3 	&= (x+y)^2(x+y)\\
% &=(x^2+2xy+y^2)(x+y)\\
% &=(x^3+2x^2y+xy^2) + (x^2y+2xy^2+y^3)\\
% &=x^3+3x^2y+3xy^2+y^3
% \end{split}					
% \end{align}

编程实现一个LSTM模型来完成中文自动切词任务,将连续的中文文本切分成词序列。通过测试数据计算Precision, Recall, F-score. 

数据集来自Sighan-2004中文切词国际比赛的标准数据.

训练数据为86924个句子,测试数据为3985个句子,提供了标准测试数据答案.

%------------------------------------------------

% \subsection{Heading on level 2 (subsection)}

% Lorem ipsum dolor sit amet, consectetuer adipiscing elit. 
% \begin{align}
% A = 
% \begin{bmatrix}
% A_{11} & A_{21} \\
% A_{21} & A_{22}
% \end{bmatrix}
% \end{align}
% Aenean commodo ligula eget dolor. Aenean massa. Cum sociis natoque penatibus et magnis dis parturient montes, nascetur ridiculus mus. Donec quam felis, ultricies nec, pellentesque eu, pretium quis, sem.

% %------------------------------------------------

% \subsubsection{Heading on level 3 (subsubsection)}

% \lipsum[3] % Dummy text

% \paragraph{Heading on level 4 (paragraph)}

% \lipsum[6] % Dummy text

%----------------------------------------------------------------------------------------
%	PROBLEM 2
%----------------------------------------------------------------------------------------

\section{实验方法}
\subsection{分词的字标注法}
通过字标注法将分词问题变成序列标注问题,对于输入序列,输出大小相同、一一对应的标注序列.

能够实现分词的最简单标注方法是2tag,即通过0、1判断切分或不切分,但通常情况下2tag表现不好,因为两个标记很难体现语义规律.
在这里选择了更常用的4tag标记,分为{b:begin, 多字词的开头; m: middle, 多字词的中间; e: end, 多字词的结尾; s: single, 单字词}.
除此之外还使用了字符'x'标记字符长度小于max$\_$length时padding的部分.
\begin{lstlisting}
我  认为  人生  价值  的  实现  应该  有  两  方面  的  内容  :  一  是  个人  对  社会  的  贡献  ,  二  是  社会  对  个人  的  尊重  和  满足  。

s be be be be s be be s s be s be s s sbe s be s be s s s be s be s be s be s
\end{lstlisting}

\subsection{训练网络,双向LSTM}
LSTM(Long SHort Term Memory)是时间递归神经网络,能解决RNN中的长期依赖问题,如图\ref{pic_lstm}所示.

\begin{figure}[htb]
    \centering
    \includegraphics[width=0.8\textwidth]{pic/LSTM.png}
    \caption{LSTM,中的重复模块包含四个交互的层}
    \label{pic_lstm}
\end{figure}

LSTM的关键就是细胞状态,水平线在图上方贯穿运行.
细胞状态类似于传送带,直接在整个链上运行,只有一些少量的线性交互.
信息在上面流传保持不变会很容易.



\subsection{Viterbi算法求解全局最优路径}
% \subsection{Example of list (3*itemize)}
% \begin{itemize}
% 	\item First item in a list 
% 		\begin{itemize}
% 		\item First item in a list 
% 			\begin{itemize}
% 			\item First item in a list 
% 			\item Second item in a list 
% 			\end{itemize}
% 		\item Second item in a list 
% 		\end{itemize}
% 	\item Second item in a list 
% \end{itemize}

%------------------------------------------------


% \subsection{Example of list (enumerate)}
% \begin{enumerate}
% \item First item in a list 
% \item Second item in a list 
% \item Third item in a list
% \end{enumerate}

%----------------------------------------------------------------------------------------

\section{实验设置与步骤}
\subsection{数据预处理}
\section{实验结果}






\end{document}